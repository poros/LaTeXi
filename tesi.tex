%CLASSE DOCUMENTO - LINGUA E DIMENSIONE FONT
\documentclass[corpo=11pt,numerazioneromana]{toptesi}

%%%%%%%%%%%%%%%%%%%%%%%%%%%%%%%%%%%%%%%%%%%%%%%%%%%%%%%%%%%%%%%

% INCLUSIONE PACCHETTI
\usepackage[classica]{topfront}
\usepackage[utf8]{inputenc} %utf8
\usepackage[italian]{babel}
\usepackage[T1]{fontenc}
\usepackage{blindtext}
\usepackage{graphicx,wrapfig}
\usepackage{booktabs}
\usepackage{lmodern}
\usepackage{varioref}
\usepackage{url}
\usepackage{array}
\usepackage{paralist}{\obeyspaces\global\let =\space}
\usepackage{verbatim}
\usepackage{subfig}
\usepackage{tabularx}
\usepackage{amsmath}
\usepackage{amsfonts}
\usepackage{float}
\usepackage{amssymb}
\usepackage{multicol}
\usepackage{multirow}
\usepackage{listings}
\usepackage[pass]{geometry}
\usepackage[figuresright]{rotating}
\usepackage{algorithm}
\usepackage{algorithmic}
\usepackage{amsmath}
\usepackage[babel]{csquotes}
\usepackage{hyperref}
\usepackage[backend=biber,bibencoding=ascii]{biblatex}


%%%%%%%%%%%%%%%%%%%%%%%%%%%%%%%%%%%%%%%%%%%%%%%%%%%%%%%%%%%%%%%

% CONFIGURAZIONE LINK E RIFERIMENTI
\hypersetup{%
    pdfpagemode={UseOutlines},
    bookmarksopen,
    pdfstartview={FitH},
    colorlinks,
    linkcolor={black}, %COLORE DEI RIFERIMENTI AL TESTO
    citecolor={blue}, %COLORE DEI RIFERIMENTI ALLE CITAZIONI
    urlcolor={blue} %COLORI DEGLI URL
}

%%%%%%%%%%%%%%%%%%%%%%%%%%%%%%%%%%%%%%%%%%%%%%%%%%%%%%%%%%%%%%%

% CONFIGURAZIONE LISTATI/CODICE - CANCELLARE SE NON NECESSARIO
% PYTHON - BIANCO E NERO
\lstset{%
	captionpos=b,
	language=Python,
	basicstyle =\small\ttfamily,
	keywordstyle=\color{black}\bfseries,
	breaklines=true,
	breakatwhitespace=true,
	frame=lines,
	numbers=left,
	numberstyle=\footnotesize,
}

%%%%%%%%%%%%%%%%%%%%%%%%%%%%%%%%%%%%%%%%%%%%%%%%%%%%%%%%%%%%%%%

% FRENCHSPACING VA _SEMPRE_ ABILITATO PER DOCUMENTI IN ITALIANO
\frenchspacing

%%%%%%%%%%%%%%%%%%%%%%%%%%%%%%%%%%%%%%%%%%%%%%%%%%%%%%%%%%%%%%%

%DEFINIZIONE SEZIONI IN NUMERAZIONE ROMANA
%ELENCO DEI LISTATI/CODICI
\makeatletter
\newcommand\listofcodes{%
 \iffrontmatter\else\frontmattertrue\fi
 \if@openright\cleardoublepage\else\clearpage\fi
 % change the meaning of \chapter in a group
 \begingroup\def\chapter##1{\@schapter}
 \phantomsection % for the hyperlink
 \addcontentsline{toc}{chapter}{Elenco dei listati}
 \lstlistoflistings
 \endgroup
}
\makeatother

\addto\captionsitalian{%
  \renewcommand{\lstlistlistingname}{Elenco dei listati}%
  \renewcommand{\lstlistingname}{Listato}%
}

%%%%%%%%%%%%%%%%%%%%%%%%%%%%%%%%%%%%%%%%%%%%%%%%%%%%%%%%%%%%%%%

% INFORMAZIONI PDF - PERSONALIZZARE
\pdfinfo{%
  /Title    (Senza carta igienica)
  /Author   (Tinaso de Tinasis)
  /Subject  (Un trattato di toccante urgenza)
  /Keywords (LaTeXi carta igienica urgenza Politecnico Presicce)
}

%%%%%%%%%%%%%%%%%%%%%%%%%%%%%%%%%%%%%%%%%%%%%%%%%%%%%%%%%%%%%%%

% LISTA DEI CAPITOLI DA INCLUDERE - PERSONALIZZARE
\includeonly{%
chap_qui,%
chap_quo,%
chap_qua,%
app_a,%
}

% FILE DI BIBLIOGRAFIA
\addbibresource{bibliography.bib}

%%%%%%%%%%%%%%%%%%%%%%%%%%%%%%%%%%%%%%%%%%%%%%%%%%%%%%%%%%%%%%%

% INIZIO DOCUMENTO
\begin{document}

%%%%%%%%%%%%%%%%%%%%%%%%%%%%%%%%%%%%%%%%%%%%%%%%%%%%%%%%%%%%%%%

% FRONTESPIZIO - PERSONALIZZARE
% ELIMINATE LE VOCI CHE NON VI SERVONO

% UNIVERSITA - NOME
\ateneo{Politecnico di Presicce}

% FACOLTA - DICITURA - CANCELLARE O DECOMMENTARE
%\FacoltaDi{Faculty of}
% FACOLTA - NOME
\facolta{Ingegneria}

% CORSO DI LAUREA - DICITURA (MANTENERE LO SPAZIO) - CANCELLARE O DECOMMENTARE
%\CorsoDiLaureaIn{Master of Science in }
% CORSO DI LAUREA - NOME
\corsodilaurea{Ortodonzia Bovina}

% TIPOLOGIA TESI
\TesiDiLaurea{Tesi di Laurea Magistrale}

% TITOLO
\titolo{Senza carta igienica}

% SOTTOTITOLO
\sottotitolo{Un trattato di toccante urgenza}

% RELATORE/I - DICITURA - CANCELLARE SE UN SOLO RELATORE
%\AdvisorName{Relatori}
% RELATORE - PROF. NOME E COGNOME
\relatore{prof.\ Darth Vader}
% RELATORE AGGIUNTIVO - PROF NOME E COGNOME
% SE SI HA SOLO UN RELATORE ELIMINARE INSIEME AD AdvisorName
\secondorelatore{prof.\ Neo Cortex}

% TUTORE AZIENDALE - TITOLO NOME E COGNOME
\tutoreaziendale{Ing. Carlino Cane}
% TUTORE AZIENDALE - DICITURA//AZIENDA
\NomeTutoreAziendale{Tutore Aziendale\\FeelGood Inc}

% CANDIDATO - DICITURA (MANTENERE I DUE PUNTI) - CANCELLARE O DECOMMENTARE
%\CandidateName{Candidate:}

% CANDIDATO - NOME E COGNOME
\candidato{Tinaso de Tinasis}
% CANDIDATA - ELIMINARE O SOSTITUIRE CON IL PRECEDENTE
%\candidata{Tinasa de Tinasis} 
% SECONDO CANDIDATO - ELIMINARE O DECOMMENTARE
%secondocandidato{Bombo de Bombis}
%secondacandidata{Bomba de Bombis}

% LOGO UNIVERSITA
\logosede{images/logo}

% DATA - MESE ANNO
\sedutadilaurea{Settembre 1984}

\frontespizio

%%%%%%%%%%%%%%%%%%%%%%%%%%%%%%%%%%%%%%%%%%%%%%%%%%%%%%%%%%%%%%%

%INTERLINEA - DEFAULT 1 - NON ESAGERATE, NON SUPERATE MAI 1.3 ;)
%\interlinea{1.2}

%%%%%%%%%%%%%%%%%%%%%%%%%%%%%%%%%%%%%%%%%%%%%%%%%%%%%%%%%%%%%%%

\frontmatter

% DEDICA - PERSONALIZZARE
% VSPACE - PROPORZIONE USATA PER CENTRATURA VERTICALE DEL TESTO
% FLUSHRIGHT - ALLINEAMENTO ORIZZONTALE A DESTRA
\vspace*{\stretch{1}}
\begin{flushright}
\noindent
All'amato me stesso
\end{flushright}
\vspace*{\stretch{6}}
\cleardoublepage


% CITAZIONE - PERSONALIZZARE
% VSPACE - PROPORZIONE USATA PER CENTRATURA VERTICALE DEL TESTO
% FLUSHRIGHT - ALLINEAMENTO ORIZZONTALE A DESTRA
\vspace*{\stretch{1}}
\begin{flushright}
\noindent
Citatemi dicendo che sono stato citato male.

\textit{Groucho Marx}
\end{flushright}
\vspace*{\stretch{6}}
\cleardoublepage

%%%%%%%%%%%%%%%%%%%%%%%%%%%%%%%%%%%%%%%%%%%%%%%%%%%%%%%%%%%%%%%

% RINGRAZIAMENTI - PERSONALIZZARE
\ringraziamenti
Grazie a Grazia, Graziella e la sorella.

%%%%%%%%%%%%%%%%%%%%%%%%%%%%%%%%%%%%%%%%%%%%%%%%%%%%%%%%%%%%%%%

% ABSTRACT - PERSONALIZZARE
\sommario
Piacere, sommario.

%%%%%%%%%%%%%%%%%%%%%%%%%%%%%%%%%%%%%%%%%%%%%%%%%%%%%%%%%%%%%%%

% INDICI - ELIMINARE GLI INDICI NON NECESSARI

% INDICE GENERALE
\tableofcontents

% INDICE DELLE FIGURE
\listoffigures

% INDICE DELLE TABELLE
\listoftables

% INDICE DEI CODICI
\listofcodes

%%%%%%%%%%%%%%%%%%%%%%%%%%%%%%%%%%%%%%%%%%%%%%%%%%%%%%%%%%%%%%%

\mainmatter

% INCLUSIONE FILE CAPITOLI - PERSONALIZZARE - TENERE COERENTE CON LISTA IN ALTO
\chapter{Qui}
\label{chap:qui}

\section{Quaraquaraqua}
\label{sec:quaqaraqua}

This is a reference to a chapter \ref{chap:quo}. This is a reference to a figure \ref{fig:doge}. This is a reference to some code \ref{lst:hello}. This is a citation \cite{famous:paper}.

\lstinputlisting[label=lst:hello, firstline=2, lastline=4, caption={I directly included a portion of a file}]{code/hello.py}

\begin{lstlisting}[language=Java, label=lst:java, caption={Some code in another language than the default one}]
public void prepare(AClass foo) {
        AnotherClass bar = new AnotherClass(foo)
}
\end{lstlisting}

\Blindtext

\begin{figure}
\begin{center}
\includegraphics[width=0.5\columnwidth]{images/doge.png}
\end{center}
\caption{This is not a figure. It's a caption.}
\label{fig:doge}
\end{figure}
\chapter{Quo}
\label{chap:quo}
% DA RIMUOVERE - LOREM IPSUM PER DIMOSTRAZIONE
\foreignlanguage{english}{\Blindtext}

\chapter{Qua}
\label{chap:qua}
\Blindtext

\appendix
% INCLUSIONE APPENDICI - - PERSONALIZZARE - TENERE COERENTE CON LISTA IN ALTO
\chapter{An appendix}
\label{app:a}
% DA RIMUOVERE - LOREM IPSUM PER DIMOSTRAZIONE
\foreignlanguage{english}{\Blindtext}


%%%%%%%%%%%%%%%%%%%%%%%%%%%%%%%%%%%%%%%%%%%%%%%%%%%%%%%%%%%%%%%

% BIBLIOGRAFIA
\phantomsection
\addcontentsline{toc}{chapter}{\refname}
\nocite{*}
\printbibliography

\end{document}
